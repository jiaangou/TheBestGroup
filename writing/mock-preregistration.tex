\documentclass[]{article}
    \usepackage{lmodern}
    \usepackage{amssymb,amsmath}
\usepackage{ifxetex,ifluatex}
\usepackage{fixltx2e} % provides \textsubscript
\ifnum 0\ifxetex 1\fi\ifluatex 1\fi=0 % if pdftex
\usepackage[T1]{fontenc}
\usepackage[utf8]{inputenc}
  \else % if luatex or xelatex
\ifxetex
\usepackage{mathspec}
\usepackage{xltxtra,xunicode}
\else
  \usepackage{fontspec}
\fi
\defaultfontfeatures{Mapping=tex-text,Scale=MatchLowercase}
\newcommand{\euro}{€}
        \fi
% use upquote if available, for straight quotes in verbatim environments
\IfFileExists{upquote.sty}{\usepackage{upquote}}{}
% use microtype if available
\IfFileExists{microtype.sty}{%
  \usepackage{microtype}
  \UseMicrotypeSet[protrusion]{basicmath} % disable protrusion for tt fonts
}{}
  \usepackage[left=2.5in,bottom=1.25in,top=1.25in,right=1in]{geometry}
  \ifxetex
\usepackage[setpagesize=false, % page size defined by xetex
            unicode=false, % unicode breaks when used with xetex
            xetex]{hyperref}
\else
  \usepackage[unicode=true]{hyperref}
\fi
\hypersetup{breaklinks=true,
bookmarks=true,
pdfauthor={},
pdftitle={LDP Mock Preregistration},
colorlinks=true,
citecolor=blue,
urlcolor=blue,
linkcolor=magenta,
pdfborder={0 0 0}}
\urlstyle{same}  % don't use monospace font for urls
\setlength{\parindent}{0pt}
\setlength{\parskip}{6pt plus 2pt minus 1pt}
\setlength{\emergencystretch}{3em}  % prevent overfull lines
\providecommand{\tightlist}{%
\setlength{\itemsep}{0pt}\setlength{\parskip}{0pt}}
\setcounter{secnumdepth}{0}

% Customization for cos_prereg
\usepackage{longtable,booktabs,threeparttable,tabularx}
\linespread{1.5}
\newcounter{question}
\setcounter{question}{0}

%%% Use protect on footnotes to avoid problems with footnotes in titles
\let\rmarkdownfootnote\footnote%
\def\footnote{\protect\rmarkdownfootnote}

%%% Change title format to be more compact
\usepackage{titling}

\def\changemargin#1#2{\list{}{\rightmargin#2\leftmargin#1}\item[]}
\let\endchangemargin=\endlist

% Create subtitle command for use in maketitle
\newcommand{\subtitle}[1]{
\posttitle{
\begin{center}\large#1\end{center}
}
}

\setlength{\droptitle}{-2em}
\title{LDP Mock Preregistration}
\pretitle{\begin{changemargin}{-8pc}{0pc} \centering\large Preregistration\\ \Huge}
\posttitle{\end{changemargin}}
  \author{
          Kirsten Bevandick\textsuperscript{1},
          Kate Colson\textsuperscript{1},
          Stefano Mezzini\textsuperscript{2},
          William Ou\textsuperscript{1},
          Julee Stewart\textsuperscript{3}          \\ \vspace{0.5cm}
              \textsuperscript{1} University of British Columbia\\
              \textsuperscript{2} University of British Columbia
(Okanagan campus)\\
              \textsuperscript{3} University of Regina      }

  \def\affdep{{"", "", "", "", ""}}%
  \def\affcity{{"", "", "", "", ""}}%
  \preauthor{\begin{changemargin}{-8pc}{0pc} \centering\large}
  \postauthor{\end{changemargin}}
\date{2021-10-06}
\predate{\begin{changemargin}{-8pc}{0pc} \centering\large\emph}
\postdate{\end{changemargin}}



% Title settings
\usepackage{titlesec}
\titleformat{\section}[display]{\bfseries\Large}{\thesection}{}{}[]
\titlespacing{\section}{0pc}{*3}{*1.5}
\titleformat{\subsection}[leftmargin]{\titlerule\bfseries\filleft}{\thesubsection}{.5em}{}
\titlespacing{\subsection}{8pc}{5ex plus .1ex minus .2ex}{1.5pc}
  

% Redefines (sub)paragraphs to behave more like sections
\ifx\paragraph\undefined\else
\let\oldparagraph\paragraph
\renewcommand{\paragraph}[1]{\oldparagraph{#1}\mbox{}}
\fi
\ifx\subparagraph\undefined\else
\let\oldsubparagraph\subparagraph
\renewcommand{\subparagraph}[1]{\oldsubparagraph{#1}\mbox{}}
\fi


\begin{document}
\maketitle
\vspace{2pc}


\newcommand\Question[2]{%
   \leavevmode\par
   \stepcounter{question}
   \noindent
   \textbf{\thequestion. #1}. #2\par}

\newcommand\Answer[1]{%
    \noindent
    \textit{Registered response}: #1\par}
    
Citations in parentheses: (Gushulak \emph{et al.}, 2021), (but also, see
Gushulak \emph{et al.}, 2021)

Citation out of parentheses: Gushulak \emph{et al.} (2021) state
that\ldots{}

See what references you can add by opening the .bib file in a RStudio
and using the name after \texttt{@\{article}
(e.g.~\texttt{@article\{gushulak\_effects\_2021}), or by running
\texttt{citr:::insert\_citation()} (requires installing \texttt{citr}).

NOTE: to knit using a bib file, you will need to install a different
version of \texttt{prereg} using
\texttt{remotes::install\_github("crsh/prereg@issue-16")}. not sure why
this happens. -- Stefano

\hypertarget{roles}{%
\section{Roles}\label{roles}}

\begin{enumerate}
\def\labelenumi{\arabic{enumi}.}
\item
  \textbf{Research question + hypothesis} (William)

  \begin{itemize}
  \item
    clearly identify the research question of interest in the
    replication?
  \item
    include at least 1 testable hypothesis?
  \end{itemize}
\item
  \textbf{Data} (Julee, Kate)

  \begin{itemize}
  \item
    Description of existing data and/or data collection procedures? (If
    existing data is included, is the reference(s) to the original data
    source included?)
  \item
    A description of the variables included in the dataset and/or to be
    included in the analysis
  \item
    A study design plan?
  \end{itemize}
\item
  \textbf{Analysis}

  \begin{itemize}
  \item
    Does the pre-registration include at least 1 example statistical
    analysis using simulated/dummy data?
  \item
    Are the simulated data informed by published data?
  \end{itemize}
\item
  \textbf{Figure} (William)

  \begin{itemize}
  \item
    Does the pre-registration include a figure?
  \item
    summarize/present the key variables in the analysis (with
    appropriate response and predictor variables)?
  \item
    Include properly labelled axes and a legend (if applicable)?
  \item
    Include a figure caption
  \end{itemize}
\item
  \textbf{Literature} (Julee)

  \begin{itemize}
  \tightlist
  \item
    Does the pre-registration include in-text citations and a
    bibliography of all studies mentioned?
  \end{itemize}
\end{enumerate}

\hypertarget{study-information}{%
\section{Study Information}\label{study-information}}

\hypertarget{title}{%
\subsection{Title}\label{title}}

LDP Mock Preregistration

\hypertarget{description}{%
\subsection{Description}\label{description}}

Enter your response here.

\hypertarget{hypotheses}{%
\subsection{Hypotheses}\label{hypotheses}}

Enter your response here.

\hypertarget{design-plan}{%
\section{Design Plan}\label{design-plan}}

\hypertarget{study-type}{%
\subsection{Study type}\label{study-type}}

\textbf{Experiment}. A researcher randomly assigns treatments to study
subjects, this includes field or lab experiments. This is also known as
an intervention experiment and includes randomized controlled trials.

\textbf{Observational Study}. Data is collected from study subjects that
are not randomly assigned to a treatment. This includes surveys, natural
experiments, and regression discontinuity designs.

\textbf{Meta-Analysis}. A systematic review of published studies.

\textbf{Other}. Please explain.

\hypertarget{blinding}{%
\subsection{Blinding}\label{blinding}}

No blinding is involved in this study.

For studies that involve human subjects, they will not know the
treatment group to which they have been assigned.

Personnel who interact directly with the study subjects (either human or
non-human subjects) will not be aware of the assigned treatments.

Personnel who analyze the data collected from the study are not aware of
the treatment applied to any given group.

\hypertarget{study-design}{%
\subsection{Study design}\label{study-design}}

Enter your response here.

\hypertarget{randomization}{%
\subsection{Randomization}\label{randomization}}

Enter your response here.

\hypertarget{sampling-plan}{%
\section{Sampling Plan}\label{sampling-plan}}

\hypertarget{existing-data}{%
\subsection{Existing data}\label{existing-data}}

\textbf{Registration prior to creation of data}. As of the date of
submission of this research plan for preregistration, the data have not
yet been collected, created, or realized.

\textbf{Registration prior to any human observation of the data}. As of
the date of submission, the data exist but have not yet been quantified,
constructed, observed, or reported by anyone - including individuals
that are not associated with the proposed study. Examples include museum
specimens that have not been measured and data that have been collected
by non-human collectors and are inaccessible.

\textbf{Registration prior to accessing the data}. As of the date of
submission, the data exist, but have not been accessed by you or your
collaborators. Commonly, this includes data that has been collected by
another researcher or institution.

\textbf{Registration prior to analysis of the data}. As of the date of
submission, the data exist and you have accessed it, though no analysis
has been conducted related to the research plan (including calculation
of summary statistics). A common situation for this scenario when a
large dataset exists that is used for many different studies over time,
or when a data set is randomly split into a sample for exploratory
analyses, and the other section of data is reserved for later
confirmatory data analysis.

\textbf{Registration following analysis of the data}. As of the date of
submission, you have accessed and analyzed some of the data relevant to
the research plan. This includes preliminary analysis of variables,
calculation of descriptive statistics, and observation of data
distributions. Please see \url{https://cos.io/prereg} for more
information.

\hypertarget{explanation-of-existing-data}{%
\subsection{Explanation of existing
data}\label{explanation-of-existing-data}}

Enter your response here.

\hypertarget{data-collection-procedures}{%
\subsection{Data collection
procedures}\label{data-collection-procedures}}

Enter your response here.

\hypertarget{sample-size}{%
\subsection{Sample size}\label{sample-size}}

Enter your response here.

\hypertarget{sample-size-rationale}{%
\subsection{Sample size rationale}\label{sample-size-rationale}}

Enter your response here.

\hypertarget{stopping-rule}{%
\subsection{Stopping rule}\label{stopping-rule}}

Enter your response here.

\hypertarget{variables}{%
\section{Variables}\label{variables}}

\hypertarget{manipulated-variables}{%
\subsection{Manipulated variables}\label{manipulated-variables}}

Enter your response here.

\hypertarget{measured-variables}{%
\subsection{Measured variables}\label{measured-variables}}

Enter your response here.

\hypertarget{indices}{%
\subsection{Indices}\label{indices}}

Enter your response here.

\hypertarget{analysis-plan}{%
\section{Analysis Plan}\label{analysis-plan}}

\hypertarget{statistical-models}{%
\subsection{Statistical models}\label{statistical-models}}

Enter your response here.

\hypertarget{transformations}{%
\subsection{Transformations}\label{transformations}}

Enter your response here.

\hypertarget{inference-criteria}{%
\subsection{Inference criteria}\label{inference-criteria}}

\hypertarget{data-exclusion}{%
\subsection{Data exclusion}\label{data-exclusion}}

Enter your response here.

\hypertarget{missing-data}{%
\subsection{Missing data}\label{missing-data}}

Enter your response here.

\hypertarget{exploratory-analyses-optional}{%
\subsection{Exploratory analyses
(optional)}\label{exploratory-analyses-optional}}

Enter your response here.

\hypertarget{other}{%
\section{Other}\label{other}}

\hypertarget{other-optional}{%
\subsection{Other (Optional)}\label{other-optional}}

Enter your response here.

\hypertarget{references}{%
\section{References}\label{references}}

\hypertarget{section}{%
\subsection{}\label{section}}

\vspace{-2pc}
\setlength{\parindent}{-0.5in}
\setlength{\leftskip}{-1in}
\setlength{\parskip}{8pt}

\noindent

\hypertarget{refs}{}
\begin{CSLReferences}{1}{0}
\leavevmode\hypertarget{ref-gushulak_effects_2021}{}%
Gushulak C.A.C., Haig H.A., Kingsbury M.V., Wissel B., Cumming B.F. \&
Leavitt P.R. (2021). Effects of spatial variation in benthic phototrophs
along a depth gradient on assessments of whole‐lake processes.
\emph{Freshwater Biology}, fwb.13820.
\url{https://doi.org/10.1111/fwb.13820}

\end{CSLReferences}

\end{document}